%!TEX encoding = UTF-8 Unicode
\documentclass[a4paper,11pt,english]{article}

\usepackage[T1]{fontenc}
\usepackage[utf8]{inputenc}
\usepackage{babel}
\usepackage{indentfirst}
\usepackage[colorlinks]{hyperref}
\usepackage{graphicx}
\usepackage{paralist}
\usepackage{amsmath,amssymb}
\usepackage{footnote}
\usepackage[inline]{enumitem}
\usepackage[binary-units]{siunitx}
\usepackage[a4paper]{geometry}

\usepackage[calc,style=default]{datetime2}
\DTMsetup{datesep={--},datetimesep=T}

\renewcommand{\labelitemii}{$\circ$}

\usepackage{listings}
\usepackage{pgfplots}
\usepackage{color}
\definecolor{dkgreen}{rgb}{0,0.6,0}
\definecolor{gray}{rgb}{0.5,0.5,0.5}
\definecolor{mauve}{rgb}{0.58,0,0.82}
\lstset{language=SQL,
  basicstyle={\ttfamily},
  belowskip=4mm,
  breakatwhitespace=true,
  breaklines=true,
  classoffset=0,
  columns=flexible,
  commentstyle=\color{dkgreen},
  frame=single,
  framexleftmargin=2em,
  xleftmargin=2em,
  keywordstyle=\color{blue},
  numbers=left, %If you do not want line numbers, set `numbers=none`
  numberstyle=\small,
  showstringspaces=false,
  stringstyle=\color{mauve},
  tabsize=4,
  morekeywords={with,quarter}
}

\title{
	\begin{center}
  		\includegraphics[width=0.125\textwidth]{./images/units-logo.pdf}\\
  		\smallskip
  		\Large {University of Trieste}\\
		\smallskip
  		\large \textit{Data Management for Big Data} Course\\
		\smallskip
		\small Academic Year 2022--2023\\
  		\rule{9cm}{.4pt}\\
		\medskip
  	\end{center}
	Data Warehouse case study
}

\author{Davide Capone\thanks{\texttt{\href{mailto:davide.capone@studenti.units.it}{davide.capone@studenti.units.it}}}
\and Sandro Junior Della Rovere\thanks{\texttt{\href{mailto:sandrojunior.dellarovere@studenti.units.it}{sandrojunior.dellarovere@studenti.units.it}}}
\and Enrico Stefanel\thanks{\texttt{\href{mailto:enrico.stefanel@studenti.units.it}{enrico.stefanel@studenti.units.it}}}}


\DTMsavenow{now}%
\DTMtozulu{now}{currzulu}
\date{\footnotesize Last update on \DTMuse{currzulu}.}
%\date{\today}

\begin{document}

\maketitle


%!TEX TS-program = pdflatex
%!TEX root = ../main.tex
%!TEX encoding = UTF-8 Unicode


\section{Introduction}

The aim of this project is to study an efficient implementation of a suite of business oriented ad-hoc queries over the public TPC-H benchmark, which can be considered as a Big Data database, that has been implemented in Postgres.

\subsection{TPC-H benchmark database}

The TPC-H benchmark is a decision support benchmark that can be downloaded from the \href{https://www.tpc.org/tpch/}{TPC official website}. The data generator lets the user specify a \textit{scale factor} in order to control the size of the resulted database. Our choices was to use a \textit{scale factor} of \num{10}, meaning that the overall database size is approximately \SI{13}{\giga\byte}.

\subsubsection{Database statistics}

The benchmark is composed by eight tables:
\begin{itemize}
	\item \texttt{CUSTOMER}, with \num{16} columns and \num{1500000} tuples (\SI{312}{\mega\byte});
	\item \texttt{LINEITEM}, with \num{32} columns and \num{59986052} tuples (\SI{11}{\giga\byte}); the main attributes that are going to be used are:
		\begin{itemize}
			\item \texttt{l\_extendedprice} (\num{1351462} distinct values, i.e.\ there is an average of \num{44} tuples with the same value, that range from \num{900.91} to \num{104949.50}),
			\item \texttt{l\_discount} (\num{11} distinct values, i.e.\ there is an average of \num{5453277} tuples with the same value, that range from \num{0.00} to \num{0.10}),
			\item \texttt{l\_returnflag} (which can assume values \texttt{A}$\rightarrow$accepted, \texttt{R}$\rightarrow$returned, \texttt{N}$\rightarrow$not yet delivered; the percentage of tuples for \texttt{A} and \texttt{R} are almost \SI{25}{\percent}, while the percentage of tuples where \texttt{l\_returnflag} is \texttt{N} is about \SI{50}{\percent}),
			\item \texttt{l\_commitdate} (\num{2466} distinct values, i.e.\ there is an average of \num{24325} tuples with the same value, that range from {1992-01-31} to {1998-10-31}),
			\item \texttt{l\_receiptdate} (\num{2555} distinct values, i.e.\ there is an average of \num{23478} tuples with the same value, that range from {1992-01-03} to {1998-12-31});
		\end{itemize}
	\item \texttt{NATION}, with \num{8} columns and \num{25} tuples (\SI{24}{\kilo\byte});
	\item \texttt{ORDERS}, with \num{18} columns and \num{1500000} tuples (\SI{2481}{\kilo\byte}); the main attributes that are going to be used are:
		\begin{itemize}
			\item \texttt{o\_orderdate} (\num{2406} distinct values, i.e.\ there is an average of \num{6234} tuples with the same value, that range from {1992-01-01} to {1998-08-02});
		\end{itemize}
	\item \texttt{PART}, with \num{18} columns and \num{2000000} tuples (\SI{363}{\mega\byte}); the main attributes that are going to be used are:
		\begin{itemize}
			\item \texttt{p\_type} (\num{150} distinct values, i.e.\ there is an average of \num{13333} tuples with the same value);
		\end{itemize}
	\item \texttt{PARTSUPP}, \num{10} columns and with \num{8000000} tuples (\SI{1535}{\mega\byte});
	\item \texttt{REGION}, \num{6} columns and with \num{5} tuples (\SI{24}{\kilo\byte});
	\item \texttt{SUPPLIER}, \num{14} columns and with \num{100000} tuples (\SI{20}{\mega\byte}).
\end{itemize}

Other attributes have been used, but statistics about them have been omitted for lack of usefulness (e.g., keys of the tables, for which the cardinality is exactly the cardinality of the corresponding table).
%!TEX TS-program = pdflatex
%!TEX root = ../main.tex
%!TEX encoding = UTF-8 Unicode


\section{Set of queries}

\subsection{Export/import revenue value}

\begin{lstlisting}
	WITH lineitem_orders AS (
		SELECT 
			l_partkey, 
			l_suppkey, 
			o_orderdate, 
			o_custkey, 
			l_extendedprice, 
			l_discount
		FROM lineitem JOIN orders ON (l_orderkey = o_orderkey)
	), customer_location AS (
		SELECT 
			c_custkey, 
			c_name, 
			n_nationkey AS c_nationkey, 
			n_name AS c_nationname, 
			r_regionkey AS c_regionkey, 
			r_name AS c_regionname 
		FROM customer 
			JOIN nation ON (c_nationkey = n_nationkey)
			JOIN region ON (n_regionkey = r_regionkey)
	), supplier_location AS (
		SELECT 
			s_suppkey, 
			s_name, 
			n_nationkey AS s_nationkey, 
			n_name AS s_nationname, 
			r_regionkey AS s_regionkey, 
			r_name AS s_regionname 
		FROM supplier 
			JOIN nation ON (s_nationkey = n_nationkey)
			JOIN region ON (n_regionkey = r_regionkey)
	), query1 AS (
		SELECT
			EXTRACT (YEAR FROM o_orderdate) AS _year,
			EXTRACT (QUARTER FROM o_orderdate) AS _quarter,
			EXTRACT (MONTH FROM o_orderdate) AS _month,
			c_regionname,
			c_nationname,
			c_name,
			s_regionname,
			s_nationname,
			s_name,
			p_type,
			SUM(l_extendedprice * (1 - l_discount)) AS revenue
		FROM lineitem_orders 
			JOIN part ON l_partkey = p_partkey
			JOIN supplier_location ON (s_suppkey = l_suppkey)
			JOIN customer_location ON (c_custkey = o_custkey)
		WHERE s_nationkey <> c_nationkey
		GROUP BY
			_year,
			_quarter,
			_month,
			c_regionkey,
			c_regionname,
			c_nationkey,
			c_nationname,
			c_custkey,
			c_name,
			s_regionkey,
			s_regionname,
			s_nationkey,
			s_nationname,
			s_suppkey,
			s_name,
			p_type
	) 
\end{lstlisting}


\subsection{Late delivery}

It is asked to retrieve the number of orders where at least one ``lineitem'' has been received later than the committed date.
The aggregation should be performed with the Month $\rightarrow$ Year roll-up, and the (Customer's) Nation $\rightarrow$ Region roll-up.

\begin{lstlisting}
    WITH lineitem_orders AS (
	SELECT
		o_orderkey, 
		l_partkey, 
		l_suppkey, 
		o_orderdate, 
		o_custkey,
		l_commitdate,
		l_receiptdate
	FROM lineitem JOIN orders ON (l_orderkey = o_orderkey)
), customer_location AS (
	SELECT 
		c_custkey, 
		n_nationkey AS c_nationkey, 
		n_name AS c_nationname, 
		r_regionkey AS c_regionkey, 
		r_name AS c_regionname 
	FROM customer 
		JOIN nation ON (c_nationkey = n_nationkey)
		JOIN region ON (n_regionkey = r_regionkey)
), query2 AS (
SELECT 
	EXTRACT(YEAR FROM o_orderdate) AS _year,
	EXTRACT(MONTH FROM o_orderdate) AS _month,
	c_regionname,
	c_nationname,
	COUNT(DISTINCT(o_orderkey)) AS orders_no
FROM lineitem_orders
	JOIN part ON l_partkey = p_partkey
	JOIN customer_location ON (c_custkey = o_custkey)
WHERE 
	l_receiptdate > l_commitdate
	-- AND _month = 1
	-- AND p_type = 'PROMO BURNISHED COPPER'
GROUP BY
	_year,
	_month,
	c_regionkey,
	c_regionname,
	c_nationkey,
	c_nationname
)
SELECT * FROM query2;
\end{lstlisting}


\subsection{Returned item loss}

It is asked to retrieve the \textit{revenue loss} for customers who might be having problems with the parts that are shipped to them, where a \textit{revenue loss} is defined as
$$ \texttt{SUM(l\_extendedprice*(1-l\_discount))}$$
for all qualifying \textit{lineitems}.

\begin{lstlisting}
WITH lineitem_orders AS (
	SELECT 
		o_orderkey, 
		l_partkey, 
		l_suppkey, 
		o_orderdate, 
		o_custkey, 
		l_extendedprice, 
		l_discount, 
		l_returnflag,
		l_commitdate,
		l_receiptdate
	FROM lineitem JOIN orders ON (l_orderkey=o_orderkey)
),
query3 AS (
SELECT
	EXTRACT(YEAR FROM o_orderdate) AS _year,
	EXTRACT(QUARTER FROM o_orderdate) AS _quarter,
	EXTRACT(MONTH FROM o_orderdate) AS _month,
	c_name,
	SUM(l_extendedprice*(1-l_discount)) AS returnloss
FROM
	lineitem_orders
	JOIN customer ON (o_custkey=c_custkey)
WHERE 
	l_returnflag='R'
	-- AND c_name='Customer#000129976'
	-- AND EXTRACT(QUARTER FROM o_orderdate) = 1
GROUP BY
	_year,
	_quarter,
	_month,
	c_custkey,
	c_name
)
SELECT * FROM query3;
\end{lstlisting}

Five independent runs of the above query obtained the following execution times: \SI{753541.744}{\milli\s}, \SI{672530.120}{\milli\s}, \SI{624276.525}{\milli\s}, \SI{615741.447}{\milli\s} and \SI{634262.713}{\milli\s}.

%!TEX TS-program = pdflatex
%!TEX root = ../main.tex
%!TEX encoding = UTF-8 Unicode


\section{Indexes design}

\dots
\input{content/03_materialisation}
%%!TEX TS-program = pdflatex
%!TEX root = ../main.tex
%!TEX encoding = UTF-8 Unicode


\section{Fragmentation}
\label{sec:fragmentation}

To conclude the testings, it has been decided to try \textit{vertical fragmentation}.

It makes sense to try such an approach, since queries only use a subset of attributes of the tables. \textit{Vertical fragmentation} is usually implemented in distributes systems but it can still be useful in this case to reduce the workload induced by the queries.

\begin{lstlisting}
---- NATION: 
----- no fragmentation, the fragment used by the queries will have 3 columns and the other 1 column only.

---- REGION:
----- no fragmentation, the fragment used by the queries will have 2 columns and the other 1 column only.

---- CUSTOMER:
CREATE TABLE IF NOT EXISTS customer_frag_1
(
    c_custkey integer NOT NULL,
    c_name character varying(25) COLLATE pg_catalog."default" NOT NULL,
    c_nationkey integer NOT NULL,
    CONSTRAINT customer_frag_1_pkey PRIMARY KEY (c_custkey),
    CONSTRAINT customer_frag_1_fk1 FOREIGN KEY (c_nationkey)
        REFERENCES nation (n_nationkey) MATCH SIMPLE
        ON UPDATE NO ACTION
        ON DELETE NO ACTION
) AS
SELECT
	c_custkey,
	c_name,
	c_nationkey
FROM customer;
\end{lstlisting}
\begin{lstlisting}
CREATE TABLE IF NOT EXISTS customer_frag_2
(
	c_custkey integer NOT NULL,
    c_address character varying(40) COLLATE pg_catalog."default" NOT NULL,
    c_phone character(15) COLLATE pg_catalog."default" NOT NULL,
    c_acctbal numeric(15,2) NOT NULL,
    c_mktsegment character(10) COLLATE pg_catalog."default" NOT NULL,
    c_comment character varying(117) COLLATE pg_catalog."default" NOT NULL,
    CONSTRAINT customer_frag_2_pkey PRIMARY KEY (c_custkey)
) AS
SELECT
	c_custkey,
	c_address,
	c_phone,
	c_acctbal,
	c_mktsegment,
	c_comment
FROM customer;
\end{lstlisting}
\begin{lstlisting}
---- SUPPLIER:
CREATE TABLE IF NOT EXISTS supplier_frag_1
(
    s_suppkey integer NOT NULL,
    s_name character(25) COLLATE pg_catalog."default" NOT NULL,
    s_nationkey integer NOT NULL,
    CONSTRAINT supplier_frag_1_pkey PRIMARY KEY (s_suppkey),
    CONSTRAINT supplier_frag_1_fk1 FOREIGN KEY (s_nationkey)
        REFERENCES nation (n_nationkey) MATCH SIMPLE
        ON UPDATE NO ACTION
        ON DELETE NO ACTION
) AS
SELECT
	s_suppkey,
	s_name,
	s_nationkey
FROM supplier;
\end{lstlisting}
\begin{lstlisting}
CREATE TABLE IF NOT EXISTS supplier_frag_2
(
    s_suppkey integer NOT NULL,
    s_address character varying(40) COLLATE pg_catalog."default" NOT NULL,
    s_phone character(15) COLLATE pg_catalog."default" NOT NULL,
    s_acctbal numeric(15,2) NOT NULL,
    s_comment character varying(101) COLLATE pg_catalog."default" NOT NULL,
    CONSTRAINT supplier_frag_2_pkey PRIMARY KEY (s_suppkey)
) AS
SELECT
	s_suppkey,
	s_address,
	s_phone,
	s_acctbal,
	s_comment
FROM supplier;

---- PARTSUPP: no fragmentation (table not used in queries)
\end{lstlisting}
\begin{lstlisting}
---- PART:
CREATE TABLE IF NOT EXISTS part_frag_1
(
    p_partkey integer NOT NULL,
    p_type character varying(25) COLLATE pg_catalog."default" NOT NULL,
    CONSTRAINT part_frag_1_pkey PRIMARY KEY (p_partkey)
) AS
SELECT
	p_partkey,
	p_type
FROM part;
\end{lstlisting}
\begin{lstlisting}
CREATE TABLE IF NOT EXISTS part_frag_2
(
    p_partkey integer NOT NULL,
    p_name character varying(55) COLLATE pg_catalog."default" NOT NULL,
    p_mfgr character(25) COLLATE pg_catalog."default" NOT NULL,
    p_brand character(10) COLLATE pg_catalog."default" NOT NULL,
    p_size integer NOT NULL,
    p_container character(10) COLLATE pg_catalog."default" NOT NULL,
    p_retailprice numeric(15,2) NOT NULL,
    p_comment character varying(23) COLLATE pg_catalog."default" NOT NULL,
    CONSTRAINT part_frag_2_pkey PRIMARY KEY (p_partkey)
) AS
SELECT 
	p_partkey,
	p_name,
	p_mfgr,
	p_brand,
	p_size,
	p_container,
	p_retailprice,
	p_comment
FROM part;
\end{lstlisting}
\begin{lstlisting}
---- ORDERS:
CREATE TABLE IF NOT EXISTS orders_frag_1
(
    o_orderkey integer NOT NULL,
    o_custkey integer NOT NULL,
    o_orderdate date NOT NULL,
    CONSTRAINT orders_frag_1_pkey PRIMARY KEY (o_orderkey),
    CONSTRAINT orders_frag_1_fk1 FOREIGN KEY (o_custkey)
        REFERENCES customer (c_custkey) MATCH SIMPLE
        ON UPDATE NO ACTION
        ON DELETE NO ACTION
) AS
SELECT
	o_orderkey,
	o_custkey,
	o_orderdate
FROM orders;
\end{lstlisting}
\begin{lstlisting}
CREATE TABLE IF NOT EXISTS orders_frag_2
(
    o_orderkey integer NOT NULL,
    o_orderstatus character(1) COLLATE pg_catalog."default" NOT NULL,
    o_totalprice numeric(15,2) NOT NULL,
    o_orderpriority character(15) COLLATE pg_catalog."default" NOT NULL,
    o_clerk character(15) COLLATE pg_catalog."default" NOT NULL,
    o_shippriority integer NOT NULL,
    o_comment character varying(79) COLLATE pg_catalog."default" NOT NULL,
    CONSTRAINT orders_frag_2_pkey PRIMARY KEY (o_orderkey)
) AS
SELECT
	o_orderkey,
	o_orderstatus,
	o_totalprice,
	o_orderpriority,
	o_clerk,
	o_shippriority,
	o_comment
FROM orders;
\end{lstlisting}
\begin{lstlisting}
---- LINEITEM:
CREATE TABLE IF NOT EXISTS lineitem_frag_1
(
    l_orderkey integer NOT NULL,
    l_partkey integer NOT NULL,
    l_suppkey integer NOT NULL,
    l_linenumber integer NOT NULL,
    l_extendedprice numeric(15,2) NOT NULL,
    l_discount numeric(15,2) NOT NULL,
    l_returnflag character(1) COLLATE pg_catalog."default" NOT NULL,
    l_commitdate date NOT NULL,
    l_receiptdate date NOT NULL,
    CONSTRAINT lineitem_frag_1_pkey PRIMARY KEY (l_orderkey, l_linenumber),
    CONSTRAINT lineitem_frag_1_fk1 FOREIGN KEY (l_orderkey)
        REFERENCES orders (o_orderkey) MATCH SIMPLE
        ON UPDATE NO ACTION
        ON DELETE NO ACTION,
    CONSTRAINT lineitem_frag_1_fk2 FOREIGN KEY (l_partkey, l_suppkey)
        REFERENCES partsupp (ps_partkey, ps_suppkey) MATCH SIMPLE
        ON UPDATE NO ACTION
        ON DELETE NO ACTION
) AS
SELECT
	l_orderkey,
	l_linenumber,
	l_partkey,
	l_suppkey,
	l_extendedprice,
	l_discount,
	l_returnflag,
	l_commitdate,
	l_receiptdate
FROM lineitem;
\end{lstlisting}
\begin{lstlisting}
CREATE TABLE IF NOT EXISTS lineitem_frag_2
(
    l_orderkey integer NOT NULL,
    l_linenumber integer NOT NULL,
    l_quantity numeric(15,2) NOT NULL,
    l_tax numeric(15,2) NOT NULL,
    l_linestatus character(1) COLLATE pg_catalog."default" NOT NULL,
    l_shipdate date NOT NULL,
    l_shipinstruct character(25) COLLATE pg_catalog."default" NOT NULL,
    l_shipmode character(10) COLLATE pg_catalog."default" NOT NULL,
    l_comment character varying(44) COLLATE pg_catalog."default" NOT NULL,
    CONSTRAINT lineitem_frag_2_pkey PRIMARY KEY (l_orderkey, l_linenumber),
    CONSTRAINT lineitem_frag_2_fk FOREIGN KEY (l_orderkey)
        REFERENCES orders (o_orderkey) MATCH SIMPLE
        ON UPDATE NO ACTION
        ON DELETE NO ACTION
) AS
SELECT
	l_orderkey,
	l_linenumber,
	l_quantity,
	l_tax,
	l_linestatus,
	l_shipdate,
	l_shipinstruct,
	l_shipmode,
	l_comment
FROM lineitem;
\end{lstlisting}

The weight of the data warehouse with fragmented tables is roughly the same as the original one, since no additional data structures have been defined and the only action performed is a physical \textit{split} of relations.

\subsubsection{Execution times}

Timings have been calculated using the queries defined in \autoref{sec:queries} by only changing tables names.

\begin{table}[!h]
\centering
\begin{tabular}{|| c | c c c c c | c c ||} 
 \hline
 Query & Run 1 & Run 2 & Run 3 & Run 4 & Run 5 & 	$\mu$ & $\sigma$ \\ [0.5ex] 
 \hline\hline
 1 & 14977 & 15803 & 15851 & 16517 & 16047 & 15839 & 558 \\ 
 \hline
 2 & 20368 & 20474 & 19809 & 20284 & 20866 & 20360 & 380 \\
 \hline
 3 & 2478 & 2150 & 2145 & 2350 & 2346 & 2294 & 147 \\
 \hline
\end{tabular}
  \caption{Query timings using fragmentation, in milliseconds.}
  \label{tab:fragmentationtimings}
\end{table}



\subsection{Indexes on fragmented tables}

Since the results shown in \autoref{tab:fragmentationtimings} are promising, it has been decided to implement the \textit{indexes} (the ones defined in \autoref{subsec:indexesdef}) on the corresponding fragments.

The total size of the data warehouse at this point is \SI{20}{\giga\byte} (again, the size constraint defined in \autoref{section:introduction} is respected).


\subsubsection{Execution times}


\begin{table}[!h]
\centering
\begin{tabular}{|| c | c c c c c | c c ||} 
 \hline
 Query & Run 1 & Run 2 & Run 3 & Run 4 & Run 5 & 	$\mu$ & $\sigma$ \\ [0.5ex] 
 \hline\hline
 1 & 24459 & 22156 & 21858 & 21829 & 21471 & 22355 & 1201 \\ 
 \hline
 2 & 45378 & 40845 & 41757 & 40048 & 40449 & 41695 & 2154 \\
 \hline
 3 & 143 & 164 & 61 & 67 & 93 & 106 & 46 \\
 \hline
\end{tabular}
  \caption{Query timings using fragmentation and indexes, in milliseconds.}
  \label{tab:fragmentationindexestimings}
\end{table}
%!TEX TS-program = pdflatex
%!TEX root = ../main.tex
%!TEX encoding = UTF-8 Unicode


\section{Conclusions}
\label{section:conclusions}


\begin{figure}[h]
\centering
\begin{tikzpicture}
      \begin{axis}[
      width=.8\textwidth,
      height=20em,
      major x tick style = transparent,
      ybar=2*\pgflinewidth,
      %bar width=25pt,
      ylabel={Execution times},
      ymajorgrids = true,
      symbolic x coords={Q1,Q2,Q3},
      xtick = data,
      scaled y ticks = true,
      yticklabel={\SI[round-mode=places, round-precision=0]{\tick}{\s}},
      enlarge x limits=.5,
      ymin=0,
      legend cell align=left,
      legend style={at={(0.5,1.05)},anchor=south},
  ]
      \addplot[style={fill=2B2A4C},error bars/.cd, y dir=both, y explicit]
          coordinates {
          (Q1,40.029) += (0,.971) -= (0,.971)
          (Q2,47.626) += (0,2.626) -= (0,2.626)
          (Q3,7.893) += (0,.915) -= (0,.915)};

      \addplot[style={fill=B31312},error bars/.cd, y dir=both, y explicit]
           coordinates {
           (Q1,15.392) += (0,.483) -= (0,.483)
           (Q2,16.153) += (0,.910) -= (0,.910)
          (Q3,14.608) += (0,.732) -= (0,.732)};

      \addplot[style={fill=EA906C},error bars/.cd, y dir=both, y explicit]
           coordinates {
           (Q1,33.472) += (0,1.556) -= (0,1.556)
           (Q2,54.506) += (0,1.100) -= (0,1.100)
           (Q3,0.067) += (0,0.025) -= (0,0.025)};

      \addplot[style={fill=EEE2DE},error bars/.cd, y dir=both, y explicit]
           coordinates {
           (Q1,21.486) += (0,0.658) -= (0,0.658)
           (Q2,24.052) += (0,1.069) -= (0,1.069)
           (Q3,0.056) += (0,0.018) -= (0,0.018)};

      \legend{Na\"{i}ve query, With materialized views, With indexes, With materialized views and indexes}
  \end{axis}
  \end{tikzpicture}
  \caption{Query timings}
  \label{fig:graph}
\end{figure}


\end{document}