%!TEX TS-program = pdflatex
%!TEX root = ../main.tex
%!TEX encoding = UTF-8 Unicode


\section{Introduction}

The aim of this project is to study an efficient implementation of a suite of business oriented ad-hoc queries over the public TPC-H benchmark, which can be considered as a Big Data database, that has been implemented in Postgres.

\subsection{TPC-H benchmark database}

The TPC-H benchmark is a decision support benchmark that can be downloaded from the \href{https://www.tpc.org/tpch/}{TPC official website}. The data generator lets the user specify a \textit{scale factor} in order to control the size of the resulted database. Our choices was to use a \textit{scale factor} of \num{10}, meaning that the overall database size is approximately \SI{13}{\giga\byte}.

\subsubsection{Database statistics}

The benchmark is composed by eight tables:
\begin{itemize}
	\item \texttt{CUSTOMER}, with \num{16} columns and \num{1500000} tuples (\SI{312}{\mega\byte});
	\item \texttt{LINEITEM}, with \num{32} columns and \num{59986052} tuples (\SI{11}{\giga\byte}); the main attributes that are going to be used are:
		\begin{itemize}
			\item \texttt{l\_extendedprice} (\num{1351462} distinct values, i.e.\ there is an average of \num{44} tuples with the same value, that range from \num{900.91} to \num{104949.50}),
			\item \texttt{l\_discount} (\num{11} distinct values, i.e.\ there is an average of \num{5453277} tuples with the same value, that range from \num{0.00} to \num{0.10}),
			\item \texttt{l\_returnflag} (which can assume values \texttt{A}$\rightarrow$accepted, \texttt{R}$\rightarrow$returned, \texttt{N}$\rightarrow$not yet delivered; the percentage of tuples for \texttt{A} and \texttt{R} are almost \SI{25}{\percent}, while the percentage of tuples where \texttt{l\_returnflag} is \texttt{N} is about \SI{50}{\percent}),
			\item \texttt{l\_commitdate} (\num{2466} distinct values, i.e.\ there is an average of \num{24325} tuples with the same value, that range from {1992-01-31} to {1998-10-31}),
			\item \texttt{l\_receiptdate} (\num{2555} distinct values, i.e.\ there is an average of \num{23478} tuples with the same value, that range from {1992-01-03} to {1998-12-31});
		\end{itemize}
	\item \texttt{NATION}, with \num{8} columns and \num{25} tuples (\SI{24}{\kilo\byte});
	\item \texttt{ORDERS}, with \num{18} columns and \num{1500000} tuples (\SI{2481}{\kilo\byte}); the main attributes that are going to be used are:
		\begin{itemize}
			\item \texttt{o\_orderdate} (\num{2406} distinct values, i.e.\ there is an average of \num{6234} tuples with the same value, that range from {1992-01-01} to {1998-08-02});
		\end{itemize}
	\item \texttt{PART}, with \num{18} columns and \num{2000000} tuples (\SI{363}{\mega\byte}); the main attributes that are going to be used are:
		\begin{itemize}
			\item \texttt{p\_type} (\num{150} distinct values, i.e.\ there is an average of \num{13333} tuples with the same value);
		\end{itemize}
	\item \texttt{PARTSUPP}, \num{10} columns and with \num{8000000} tuples (\SI{1535}{\mega\byte});
	\item \texttt{REGION}, \num{6} columns and with \num{5} tuples (\SI{24}{\kilo\byte});
	\item \texttt{SUPPLIER}, \num{14} columns and with \num{100000} tuples (\SI{20}{\mega\byte}).
\end{itemize}

Other attributes have been used, but statistics about them have been omitted for lack of usefulness (e.g., keys of the tables, for which the cardinality is exactly the cardinality of the corresponding table).

\subsubsection{Database SQL definition}

The SQL definition of the tables can be found on the official benchmark download.