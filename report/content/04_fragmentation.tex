%!TEX TS-program = pdflatex
%!TEX root = ../main.tex
%!TEX encoding = UTF-8 Unicode


\section{Fragmentation}
\label{sec:fragmentation}

To conclude the testings, it has been decided to try vertical fragmentation.

It makes sense to try such an approach, since queries only use a subset of attributes of the tables. Vertical fragmentation is usually implemented in distributes systems but it can still be useful in this case to reduce the workload induced by the queries.

\begin{lstlisting}
---- NATION: 
----- no fragmentation, the fragment used by the queries will have 3 columns and the other 1 column only.

---- REGION:
----- no fragmentation, the fragment used by the queries will have 2 columns and the other 1 column only.

---- CUSTOMER:
CREATE TABLE IF NOT EXISTS customer_frag_1
(
    c_custkey integer NOT NULL,
    c_name character varying(25) COLLATE pg_catalog."default" NOT NULL,
    c_nationkey integer NOT NULL,
    CONSTRAINT customer_frag_1_pkey PRIMARY KEY (c_custkey),
    CONSTRAINT customer_frag_1_fk1 FOREIGN KEY (c_nationkey)
        REFERENCES nation (n_nationkey) MATCH SIMPLE
        ON UPDATE NO ACTION
        ON DELETE NO ACTION
) AS
SELECT
	c_custkey,
	c_name,
	c_nationkey
FROM customer;
\end{lstlisting}
\begin{lstlisting}
CREATE TABLE IF NOT EXISTS customer_frag_2
(
	c_custkey integer NOT NULL,
    c_address character varying(40) COLLATE pg_catalog."default" NOT NULL,
    c_phone character(15) COLLATE pg_catalog."default" NOT NULL,
    c_acctbal numeric(15,2) NOT NULL,
    c_mktsegment character(10) COLLATE pg_catalog."default" NOT NULL,
    c_comment character varying(117) COLLATE pg_catalog."default" NOT NULL,
    CONSTRAINT customer_frag_2_pkey PRIMARY KEY (c_custkey)
) AS
SELECT
	c_custkey,
	c_address,
	c_phone,
	c_acctbal,
	c_mktsegment,
	c_comment
FROM customer;
\end{lstlisting}
\begin{lstlisting}
---- SUPPLIER:
CREATE TABLE IF NOT EXISTS supplier_frag_1
(
    s_suppkey integer NOT NULL,
    s_name character(25) COLLATE pg_catalog."default" NOT NULL,
    s_nationkey integer NOT NULL,
    CONSTRAINT supplier_frag_1_pkey PRIMARY KEY (s_suppkey),
    CONSTRAINT supplier_frag_1_fk1 FOREIGN KEY (s_nationkey)
        REFERENCES nation (n_nationkey) MATCH SIMPLE
        ON UPDATE NO ACTION
        ON DELETE NO ACTION
) AS
SELECT
	s_suppkey,
	s_name,
	s_nationkey
FROM supplier;
\end{lstlisting}
\begin{lstlisting}
CREATE TABLE IF NOT EXISTS supplier_frag_2
(
    s_suppkey integer NOT NULL,
    s_address character varying(40) COLLATE pg_catalog."default" NOT NULL,
    s_phone character(15) COLLATE pg_catalog."default" NOT NULL,
    s_acctbal numeric(15,2) NOT NULL,
    s_comment character varying(101) COLLATE pg_catalog."default" NOT NULL,
    CONSTRAINT supplier_frag_2_pkey PRIMARY KEY (s_suppkey)
) AS
SELECT
	s_suppkey,
	s_address,
	s_phone,
	s_acctbal,
	s_comment
FROM supplier;

---- PARTSUPP: no fragmentation (table not used in queries)
\end{lstlisting}
\begin{lstlisting}
---- PART:
CREATE TABLE IF NOT EXISTS part_frag_1
(
    p_partkey integer NOT NULL,
    p_type character varying(25) COLLATE pg_catalog."default" NOT NULL,
    CONSTRAINT part_frag_1_pkey PRIMARY KEY (p_partkey)
) AS
SELECT
	p_partkey,
	p_type
FROM part;
\end{lstlisting}
\begin{lstlisting}
CREATE TABLE IF NOT EXISTS part_frag_2
(
    p_partkey integer NOT NULL,
    p_name character varying(55) COLLATE pg_catalog."default" NOT NULL,
    p_mfgr character(25) COLLATE pg_catalog."default" NOT NULL,
    p_brand character(10) COLLATE pg_catalog."default" NOT NULL,
    p_size integer NOT NULL,
    p_container character(10) COLLATE pg_catalog."default" NOT NULL,
    p_retailprice numeric(15,2) NOT NULL,
    p_comment character varying(23) COLLATE pg_catalog."default" NOT NULL,
    CONSTRAINT part_frag_2_pkey PRIMARY KEY (p_partkey)
) AS
SELECT 
	p_partkey,
	p_name,
	p_mfgr,
	p_brand,
	p_size,
	p_container,
	p_retailprice,
	p_comment
FROM part;
\end{lstlisting}
\begin{lstlisting}
---- ORDERS:
CREATE TABLE IF NOT EXISTS orders_frag_1
(
    o_orderkey integer NOT NULL,
    o_custkey integer NOT NULL,
    o_orderdate date NOT NULL,
    CONSTRAINT orders_frag_1_pkey PRIMARY KEY (o_orderkey),
    CONSTRAINT orders_frag_1_fk1 FOREIGN KEY (o_custkey)
        REFERENCES customer (c_custkey) MATCH SIMPLE
        ON UPDATE NO ACTION
        ON DELETE NO ACTION
) AS
SELECT
	o_orderkey,
	o_custkey,
	o_orderdate
FROM orders;
\end{lstlisting}
\begin{lstlisting}
CREATE TABLE IF NOT EXISTS orders_frag_2
(
    o_orderkey integer NOT NULL,
    o_orderstatus character(1) COLLATE pg_catalog."default" NOT NULL,
    o_totalprice numeric(15,2) NOT NULL,
    o_orderpriority character(15) COLLATE pg_catalog."default" NOT NULL,
    o_clerk character(15) COLLATE pg_catalog."default" NOT NULL,
    o_shippriority integer NOT NULL,
    o_comment character varying(79) COLLATE pg_catalog."default" NOT NULL,
    CONSTRAINT orders_frag_2_pkey PRIMARY KEY (o_orderkey)
) AS
SELECT
	o_orderkey,
	o_orderstatus,
	o_totalprice,
	o_orderpriority,
	o_clerk,
	o_shippriority,
	o_comment
FROM orders;
\end{lstlisting}
\begin{lstlisting}
---- LINEITEM:
CREATE TABLE IF NOT EXISTS lineitem_frag_1
(
    l_orderkey integer NOT NULL,
    l_partkey integer NOT NULL,
    l_suppkey integer NOT NULL,
    l_linenumber integer NOT NULL,
    l_extendedprice numeric(15,2) NOT NULL,
    l_discount numeric(15,2) NOT NULL,
    l_returnflag character(1) COLLATE pg_catalog."default" NOT NULL,
    l_commitdate date NOT NULL,
    l_receiptdate date NOT NULL,
    CONSTRAINT lineitem_frag_1_pkey PRIMARY KEY (l_orderkey, l_linenumber),
    CONSTRAINT lineitem_frag_1_fk1 FOREIGN KEY (l_orderkey)
        REFERENCES orders (o_orderkey) MATCH SIMPLE
        ON UPDATE NO ACTION
        ON DELETE NO ACTION,
    CONSTRAINT lineitem_frag_1_fk2 FOREIGN KEY (l_partkey, l_suppkey)
        REFERENCES partsupp (ps_partkey, ps_suppkey) MATCH SIMPLE
        ON UPDATE NO ACTION
        ON DELETE NO ACTION
) AS
SELECT
	l_orderkey,
	l_linenumber,
	l_partkey,
	l_suppkey,
	l_extendedprice,
	l_discount,
	l_returnflag,
	l_commitdate,
	l_receiptdate
FROM lineitem;
\end{lstlisting}
\begin{lstlisting}
CREATE TABLE IF NOT EXISTS lineitem_frag_2
(
    l_orderkey integer NOT NULL,
    l_linenumber integer NOT NULL,
    l_quantity numeric(15,2) NOT NULL,
    l_tax numeric(15,2) NOT NULL,
    l_linestatus character(1) COLLATE pg_catalog."default" NOT NULL,
    l_shipdate date NOT NULL,
    l_shipinstruct character(25) COLLATE pg_catalog."default" NOT NULL,
    l_shipmode character(10) COLLATE pg_catalog."default" NOT NULL,
    l_comment character varying(44) COLLATE pg_catalog."default" NOT NULL,
    CONSTRAINT lineitem_frag_2_pkey PRIMARY KEY (l_orderkey, l_linenumber),
    CONSTRAINT lineitem_frag_2_fk FOREIGN KEY (l_orderkey)
        REFERENCES orders (o_orderkey) MATCH SIMPLE
        ON UPDATE NO ACTION
        ON DELETE NO ACTION
) AS
SELECT
	l_orderkey,
	l_linenumber,
	l_quantity,
	l_tax,
	l_linestatus,
	l_shipdate,
	l_shipinstruct,
	l_shipmode,
	l_comment
FROM lineitem;
\end{lstlisting}

The weight of the data warehouse with fragmented tables is roughly the same as the original one, since no additional data structures have been defined and the only action performed is a physical \textit{split} of relations.

\subsubsection{Execution times}

Timings have been calculated using the queries defined in \autoref{sec:queries} by only changing tables names.

\begin{table}[!h]
\centering
\begin{tabular}{|| c | c c c c c | c c ||} 
 \hline
 Query & Run 1 & Run 2 & Run 3 & Run 4 & Run 5 & 	$\mu$ & $\sigma$ \\ [0.5ex] 
 \hline\hline
 1 & 14977 & 15803 & 15851 & 16517 & 16047 & 15839 & 558 \\ 
 \hline
 2 & 20368 & 20474 & 19809 & 20284 & 20866 & 20360 & 380 \\
 \hline
 3 & 2478 & 2150 & 2145 & 2350 & 2346 & 2294 & 147 \\
 \hline
\end{tabular}
  \caption{Query timings using fragmentation, in milliseconds.}
  \label{tab:fragmentationtimings}
\end{table}



\subsection{Indexes on fragmented tables}

Since the results shown in \autoref{tab:fragmentationtimings} are promising, it has been decided to implement the indexes (the ones defined in \autoref{subsec:indexesdef}) on the corresponding fragments.

The total size of the data warehouse at this point is \SI{20}{\giga\byte} (again, the size constraint defined in \autoref{section:introduction} is respected).


\subsubsection{Execution times}


\begin{table}[!h]
\centering
\begin{tabular}{|| c | c c c c c | c c ||} 
 \hline
 Query & Run 1 & Run 2 & Run 3 & Run 4 & Run 5 & 	$\mu$ & $\sigma$ \\ [0.5ex] 
 \hline\hline
 1 & 24459 & 22156 & 21858 & 21829 & 21471 & 22355 & 1201 \\ 
 \hline
 2 & 45378 & 40845 & 41757 & 40048 & 40449 & 41695 & 2154 \\
 \hline
 3 & 143 & 164 & 61 & 67 & 93 & 106 & 46 \\
 \hline
\end{tabular}
  \caption{Query timings using fragmentation and indexes, in milliseconds.}
  \label{tab:fragmentationindexestimings}
\end{table}