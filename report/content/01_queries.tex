%!TEX TS-program = pdflatex
%!TEX root = ../main.tex
%!TEX encoding = UTF-8 Unicode


\section{Set of queries}

\subsection{Export/import revenue value}

It is asked to return the \textit{export/import revenue} between two different nations (E, I) where E is the nation of the lineitem supplier 
and I the nation of the lineitem customer, and where the \textit{revenue} is defined as $$ \texttt{SUM(l\_extendedprice*(1-l\_discount))}$$.
The aggregation should be performed with the Month $\rightarrow$ Quarter $\rightarrow$ Year, (Part) Type and Nation $\rightarrow$ Region roll-ups.

The query is implemented in such a way that it allows the slicing over (Part) Type and Exporting Nation.

\begin{lstlisting}
WITH lineitem_orders AS (
	SELECT 
		l_partkey, 
		l_suppkey, 
		o_orderdate, 
		o_custkey, 
		l_extendedprice, 
		l_discount
	FROM lineitem JOIN orders ON (l_orderkey = o_orderkey)
), customer_location AS (
	SELECT 
		c_custkey, 
		c_name, 
		n_nationkey AS c_nationkey, 
		n_name AS c_nationname, 
		r_regionkey AS c_regionkey, 
		r_name AS c_regionname 
	FROM customer 
		JOIN nation ON (c_nationkey = n_nationkey)
		JOIN region ON (n_regionkey = r_regionkey)
), supplier_location AS (
	SELECT 
		s_suppkey, 
		s_name, 
		n_nationkey AS s_nationkey, 
		n_name AS s_nationname, 
		r_regionkey AS s_regionkey, 
		r_name AS s_regionname 
	FROM supplier 
		JOIN nation ON (s_nationkey = n_nationkey)
		JOIN region ON (n_regionkey = r_regionkey)
), query1 AS (
	SELECT
		EXTRACT (YEAR FROM o_orderdate) AS _year,
		EXTRACT (QUARTER FROM o_orderdate) AS _quarter,
		EXTRACT (MONTH FROM o_orderdate) AS _month,
		c_regionname,
		c_nationname,
		c_name,
		s_regionname,
		s_nationname,
		s_name,
		p_type,
		SUM(l_extendedprice * (1 - l_discount)) AS revenue
	FROM lineitem_orders 
		JOIN part ON l_partkey = p_partkey
		JOIN supplier_location ON (s_suppkey = l_suppkey)
		JOIN customer_location ON (c_custkey = o_custkey)
	WHERE
		s_nationkey <> c_nationkey
		-- AND p_type = 'PROMO BURNISHED COPPER'
		-- AND s_nationname = 'UNITED STATES'
	GROUP BY
		_year,
		_quarter,
		_month,
		c_regionkey,
		c_regionname,
		c_nationkey,
		c_nationname,
		c_custkey,
		c_name,
		s_regionkey,
		s_regionname,
		s_nationkey,
		s_nationname,
		s_suppkey,
		s_name,
		p_type
)
SELECT * FROM query1;
\end{lstlisting}


\subsection{Late delivery}

It is asked to retrieve the number of orders where at least one ``lineitem'' has been received later than the committed date.

The aggregation should be performed with the Month $\rightarrow$ Year roll-up, and the (Customer's) Nation $\rightarrow$ Region roll-up.

The query has been implemented in such a way that it allows the slicing over a specific Month, and a specific (Part) Type.

\begin{lstlisting}
WITH lineitem_orders AS (
	SELECT
		o_orderkey, 
		l_partkey, 
		l_suppkey, 
		o_orderdate, 
		o_custkey,
		l_commitdate,
		l_receiptdate
	FROM lineitem JOIN orders ON (l_orderkey = o_orderkey)
), customer_location AS (
	SELECT 
		c_custkey, 
		n_nationkey AS c_nationkey, 
		n_name AS c_nationname, 
		r_regionkey AS c_regionkey, 
		r_name AS c_regionname 
	FROM customer 
		JOIN nation ON (c_nationkey = n_nationkey)
		JOIN region ON (n_regionkey = r_regionkey)
), query2 AS (
SELECT 
	EXTRACT(YEAR FROM o_orderdate) AS _year,
	EXTRACT(MONTH FROM o_orderdate) AS _month,
	c_regionname,
	c_nationname,
	COUNT(DISTINCT(o_orderkey)) AS orders_no
FROM lineitem_orders
	JOIN part ON l_partkey = p_partkey
	JOIN customer_location ON (c_custkey = o_custkey)
WHERE 
	l_receiptdate > l_commitdate
	-- AND _month = 1
	-- AND p_type = 'PROMO BURNISHED COPPER'
GROUP BY
	_year,
	_month,
	c_regionkey,
	c_regionname,
	c_nationkey,
	c_nationname
)
SELECT * FROM query2;
\end{lstlisting}


\subsection{Returned item loss}

It is asked to retrieve the \textit{revenue loss} for customers who might be having problems with the parts that are shipped to them, where a \textit{revenue loss} is defined as
$$ \texttt{SUM(l\_extendedprice*(1-l\_discount))}$$
for all qualifying \textit{lineitems}.

The aggregations should be performed with the Month $\rightarrow$ Quarter $\rightarrow$ Year and Customer roll-ups.

The query has been implemented in such a way that it allows the slicing over the Name of a Customer combined with a specific Quarter.

\begin{lstlisting}
WITH lineitem_orders AS (
	SELECT 
		o_orderkey, 
		o_orderdate, 
		o_custkey, 
		l_extendedprice, 
		l_discount, 
		l_returnflag
	FROM lineitem JOIN orders ON (l_orderkey=o_orderkey)
),
query3 AS (
SELECT
	EXTRACT(YEAR FROM o_orderdate) AS _year,
	EXTRACT(QUARTER FROM o_orderdate) AS _quarter,
	EXTRACT(MONTH FROM o_orderdate) AS _month,
	c_name,
	SUM(l_extendedprice*(1-l_discount)) AS returnloss
FROM
	lineitem_orders
	JOIN customer ON (o_custkey=c_custkey)
WHERE 
	l_returnflag='R'
	-- AND c_name='Customer#000129976'
	-- AND EXTRACT(QUARTER FROM o_orderdate) = 1
GROUP BY
	_year,
	_quarter,
	_month,
	c_custkey,
	c_name
)
SELECT * FROM query3;
\end{lstlisting}

\subsection{Execution times}

The query timings have been measured as previously discussed in \autoref{subsection:organizationwork}.
Furthermore, no consecutive runs have been performed on the same query, in order to reduce external biases (following a Round-Robin schema).

\begin{table}[!h]
\centering
\begin{tabular}{|| c | c c c c c | c c ||} 
 \hline
 Query & Run 1 & Run 2 & Run 3 & Run 4 & Run 5 & 	$\mu$ & $\sigma$ \\ [0.5ex] 
 \hline\hline
 1 & 643 & 685 & 724 & 698 & 813 & 713 & 63 \\ 
 \hline
 2 & 201 & 203 & 205 & 203 & 203 & 203 & 2 \\
 \hline
 3 & 98 & 102 & 101 & 102 & 101 & 101 & 2 \\
 \hline
\end{tabular}
  \caption{Na\"{i}ve query timings, in seconds.}
  \label{tab:vanillatimings}
\end{table}