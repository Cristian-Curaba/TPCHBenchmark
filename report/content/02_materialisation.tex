%!TEX TS-program = pdflatex
%!TEX root = ../main.tex
%!TEX encoding = UTF-8 Unicode


\section{Materialization}
\label{section:materialization}

After a study on the given set of queries, some common intermediate results have been detected between the three. In order to try lowering the average execution cost, materialized views have been defined starting from the said results.

Furthermore, since the problem considers dealing with a data warehouse (OLAP), there won't be frequent updates, so materialized views are a smart choice.

\subsection{Lineitem - Orders View}
Since all the queries perform a join between relations \textit{lineitem} and \textit{orders}, an idea is to pre-process this join by creating a materialized view.

\begin{lstlisting}
CREATE MATERIALIZED VIEW lineitem_orders_mv AS
	SELECT 
		o_orderkey, 
		l_partkey, 
		l_suppkey, 
		o_orderdate, 
		o_custkey, 
		l_extendedprice, 
		l_discount, 
		l_returnflag,
		l_commitdate,
		l_receiptdate
	FROM lineitem JOIN orders ON (l_orderkey = o_orderkey);
\end{lstlisting}

This materialized view is composed by \num{59986052} rows, and weighs \SI{4378}{\mega\byte}.

\subsection{Customer - Location View}
Queries 1 and 2 execute a join operation between \textit{customer}, \textit{nation} and \textit{region}, also this step has been pre-processed by creating a materialized view.

\begin{lstlisting}
CREATE MATERIALIZED VIEW customer_location_mv AS
	SELECT 
		c_custkey, 
		c_name, 
		n_nationkey AS c_nationkey, 
		n_name AS c_nationname, 
		r_regionkey AS c_regionkey, 
		r_name AS c_regionname 
	FROM customer 
		JOIN nation ON (c_nationkey = n_nationkey)
		JOIN region ON (n_regionkey = r_regionkey);
\end{lstlisting}

This materialized view is composed by \num{1500000} rows, and weighs \SI{167}{\mega\byte}.

\subsection{Supplier - Location View}
Finally, queries 1 and 2 execute a join operation between \textit{supplier}, \textit{nation} and \textit{region}, also this step has been pre-processed by creating a materialized view.

\begin{lstlisting}
CREATE MATERIALIZED VIEW supplier_location_mv AS
	SELECT 
		s_suppkey, 
		s_name, 
		n_nationkey AS s_nationkey, 
		n_name AS s_nationname, 
		r_regionkey AS s_regionkey, 
		r_name AS s_regionname 
	FROM supplier 
		JOIN nation ON (s_nationkey = n_nationkey)
		JOIN region ON (n_regionkey = r_regionkey);
\end{lstlisting}

This materialized view is composed by \num{100000} rows, and weighs \SI{12}{\mega\byte}.

\subsection{Queries with materialized views}
\label{subsection:queriesmaterilizedviews}

\begin{lstlisting}
WITH query1 AS (
	SELECT
		EXTRACT(YEAR FROM o_orderdate) AS _year,
		EXTRACT(QUARTER FROM o_orderdate) AS _quarter,
		EXTRACT(MONTH FROM o_orderdate) AS _month,
		c_regionname,
		c_nationname,
		c_name,
		s_regionname,
		s_nationname,
		s_name,
		p_type,
		SUM(l_extendedprice * (1 - l_discount)) AS revenue
	FROM lineitem_orders_mv 
		JOIN part ON l_partkey = p_partkey
		JOIN supplier_location_mv ON (s_suppkey = l_suppkey)
		JOIN customer_location_mv ON (c_custkey = o_custkey)
	WHERE
		s_nationkey <> c_nationkey
		AND p_type = 'PROMO BURNISHED COPPER'
		AND s_nationname = 'UNITED STATES'
	GROUP BY
		_year,
		_quarter,
		_month,
		c_regionkey,
		c_regionname,
		c_nationkey,
		c_nationname,
		c_custkey,
		c_name,
		s_regionkey,
		s_regionname,
		s_nationkey,
		s_nationname,
		s_suppkey,
		s_name,
		p_type
)
SELECT * FROM query1;
\end{lstlisting}

\begin{lstlisting}
WITH query2 AS (
SELECT 
	EXTRACT(YEAR FROM o_orderdate) AS _year,
	EXTRACT(MONTH FROM o_orderdate) AS _month,
	c_regionname,
	c_nationname,
	COUNT(DISTINCT(o_orderkey)) AS orders_no
FROM lineitem_orders_mv
	JOIN part ON l_partkey = p_partkey
	JOIN customer_location_mv ON (c_custkey = o_custkey)
WHERE 
	l_receiptdate > l_commitdate
	AND EXTRACT(MONTH FROM o_orderdate) = 1
	AND p_type = 'PROMO BURNISHED COPPER'
GROUP BY
	_year,
	_month,
	c_regionkey,
	c_regionname,
	c_nationkey,
	c_nationname
)
SELECT * FROM query2;
\end{lstlisting}

\begin{lstlisting}
WITH query3 AS (
SELECT
	EXTRACT(YEAR FROM o_orderdate) AS _year,
	EXTRACT(QUARTER FROM o_orderdate) AS _quarter,
	EXTRACT(MONTH FROM o_orderdate) AS _month,
	c_name,
	SUM(l_extendedprice * (1 - l_discount)) AS returnloss
FROM
	lineitem_orders_mv
	JOIN customer ON o_custkey = c_custkey
WHERE 
	l_returnflag = 'R'
	AND c_name = 'Customer#000129976'
	AND EXTRACT(QUARTER FROM o_orderdate) = 1
GROUP BY
	_year,
	_quarter,
	_month,
	c_custkey,
	c_name
)
SELECT * FROM query3;
\end{lstlisting}

\subsection{Execution times}

As for the base versions of queries, we collected statistics on execution timings and results can be observed on \autoref{tab:materializationstimings}. Further considerations are reported in the \autoref{section:conclusions}.

\begin{table}[!h]
\centering
\begin{tabular}{|| c | c c c c c | c c ||} 
 \hline
 Query & Run 1 & Run 2 & Run 3 & Run 4 & Run 5 & 	$\mu$ & $\sigma$ \\ [0.5ex] 
 \hline\hline
 1 & \num{16207} & \num{15048} & \num{15257} & \num{15421} & \num{15028} & \num{15392} & \num{483} \\ 
 \hline
 2 & \num{14957} & \num{17235} & \num{15519} & \num{16654} & \num{16400} & \num{16153} & \num{910} \\ 
 \hline
 3 & \num{13742} & \num{15604} & \num{14050} & \num{14822} & \num{14822} & \num{14608} & \num{732} \\ 
 \hline
\end{tabular}
  \caption{Query timings with materialized views, in milliseconds.}
  \label{tab:materializationstimings}
\end{table}